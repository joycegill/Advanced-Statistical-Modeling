% Options for packages loaded elsewhere
\PassOptionsToPackage{unicode}{hyperref}
\PassOptionsToPackage{hyphens}{url}
%
\documentclass[
]{article}
\usepackage{amsmath,amssymb}
\usepackage{iftex}
\ifPDFTeX
  \usepackage[T1]{fontenc}
  \usepackage[utf8]{inputenc}
  \usepackage{textcomp} % provide euro and other symbols
\else % if luatex or xetex
  \usepackage{unicode-math} % this also loads fontspec
  \defaultfontfeatures{Scale=MatchLowercase}
  \defaultfontfeatures[\rmfamily]{Ligatures=TeX,Scale=1}
\fi
\usepackage{lmodern}
\ifPDFTeX\else
  % xetex/luatex font selection
\fi
% Use upquote if available, for straight quotes in verbatim environments
\IfFileExists{upquote.sty}{\usepackage{upquote}}{}
\IfFileExists{microtype.sty}{% use microtype if available
  \usepackage[]{microtype}
  \UseMicrotypeSet[protrusion]{basicmath} % disable protrusion for tt fonts
}{}
\makeatletter
\@ifundefined{KOMAClassName}{% if non-KOMA class
  \IfFileExists{parskip.sty}{%
    \usepackage{parskip}
  }{% else
    \setlength{\parindent}{0pt}
    \setlength{\parskip}{6pt plus 2pt minus 1pt}}
}{% if KOMA class
  \KOMAoptions{parskip=half}}
\makeatother
\usepackage{xcolor}
\usepackage[margin=1in]{geometry}
\usepackage{color}
\usepackage{fancyvrb}
\newcommand{\VerbBar}{|}
\newcommand{\VERB}{\Verb[commandchars=\\\{\}]}
\DefineVerbatimEnvironment{Highlighting}{Verbatim}{commandchars=\\\{\}}
% Add ',fontsize=\small' for more characters per line
\usepackage{framed}
\definecolor{shadecolor}{RGB}{248,248,248}
\newenvironment{Shaded}{\begin{snugshade}}{\end{snugshade}}
\newcommand{\AlertTok}[1]{\textcolor[rgb]{0.94,0.16,0.16}{#1}}
\newcommand{\AnnotationTok}[1]{\textcolor[rgb]{0.56,0.35,0.01}{\textbf{\textit{#1}}}}
\newcommand{\AttributeTok}[1]{\textcolor[rgb]{0.13,0.29,0.53}{#1}}
\newcommand{\BaseNTok}[1]{\textcolor[rgb]{0.00,0.00,0.81}{#1}}
\newcommand{\BuiltInTok}[1]{#1}
\newcommand{\CharTok}[1]{\textcolor[rgb]{0.31,0.60,0.02}{#1}}
\newcommand{\CommentTok}[1]{\textcolor[rgb]{0.56,0.35,0.01}{\textit{#1}}}
\newcommand{\CommentVarTok}[1]{\textcolor[rgb]{0.56,0.35,0.01}{\textbf{\textit{#1}}}}
\newcommand{\ConstantTok}[1]{\textcolor[rgb]{0.56,0.35,0.01}{#1}}
\newcommand{\ControlFlowTok}[1]{\textcolor[rgb]{0.13,0.29,0.53}{\textbf{#1}}}
\newcommand{\DataTypeTok}[1]{\textcolor[rgb]{0.13,0.29,0.53}{#1}}
\newcommand{\DecValTok}[1]{\textcolor[rgb]{0.00,0.00,0.81}{#1}}
\newcommand{\DocumentationTok}[1]{\textcolor[rgb]{0.56,0.35,0.01}{\textbf{\textit{#1}}}}
\newcommand{\ErrorTok}[1]{\textcolor[rgb]{0.64,0.00,0.00}{\textbf{#1}}}
\newcommand{\ExtensionTok}[1]{#1}
\newcommand{\FloatTok}[1]{\textcolor[rgb]{0.00,0.00,0.81}{#1}}
\newcommand{\FunctionTok}[1]{\textcolor[rgb]{0.13,0.29,0.53}{\textbf{#1}}}
\newcommand{\ImportTok}[1]{#1}
\newcommand{\InformationTok}[1]{\textcolor[rgb]{0.56,0.35,0.01}{\textbf{\textit{#1}}}}
\newcommand{\KeywordTok}[1]{\textcolor[rgb]{0.13,0.29,0.53}{\textbf{#1}}}
\newcommand{\NormalTok}[1]{#1}
\newcommand{\OperatorTok}[1]{\textcolor[rgb]{0.81,0.36,0.00}{\textbf{#1}}}
\newcommand{\OtherTok}[1]{\textcolor[rgb]{0.56,0.35,0.01}{#1}}
\newcommand{\PreprocessorTok}[1]{\textcolor[rgb]{0.56,0.35,0.01}{\textit{#1}}}
\newcommand{\RegionMarkerTok}[1]{#1}
\newcommand{\SpecialCharTok}[1]{\textcolor[rgb]{0.81,0.36,0.00}{\textbf{#1}}}
\newcommand{\SpecialStringTok}[1]{\textcolor[rgb]{0.31,0.60,0.02}{#1}}
\newcommand{\StringTok}[1]{\textcolor[rgb]{0.31,0.60,0.02}{#1}}
\newcommand{\VariableTok}[1]{\textcolor[rgb]{0.00,0.00,0.00}{#1}}
\newcommand{\VerbatimStringTok}[1]{\textcolor[rgb]{0.31,0.60,0.02}{#1}}
\newcommand{\WarningTok}[1]{\textcolor[rgb]{0.56,0.35,0.01}{\textbf{\textit{#1}}}}
\usepackage{graphicx}
\makeatletter
\def\maxwidth{\ifdim\Gin@nat@width>\linewidth\linewidth\else\Gin@nat@width\fi}
\def\maxheight{\ifdim\Gin@nat@height>\textheight\textheight\else\Gin@nat@height\fi}
\makeatother
% Scale images if necessary, so that they will not overflow the page
% margins by default, and it is still possible to overwrite the defaults
% using explicit options in \includegraphics[width, height, ...]{}
\setkeys{Gin}{width=\maxwidth,height=\maxheight,keepaspectratio}
% Set default figure placement to htbp
\makeatletter
\def\fps@figure{htbp}
\makeatother
\setlength{\emergencystretch}{3em} % prevent overfull lines
\providecommand{\tightlist}{%
  \setlength{\itemsep}{0pt}\setlength{\parskip}{0pt}}
\setcounter{secnumdepth}{-\maxdimen} % remove section numbering
\ifLuaTeX
  \usepackage{selnolig}  % disable illegal ligatures
\fi
\usepackage{bookmark}
\IfFileExists{xurl.sty}{\usepackage{xurl}}{} % add URL line breaks if available
\urlstyle{same}
\hypersetup{
  pdftitle={Data Cleaning},
  pdfauthor={Khanh Do, Joyce Gill},
  hidelinks,
  pdfcreator={LaTeX via pandoc}}

\title{Data Cleaning}
\author{Khanh Do, Joyce Gill}
\date{2026-01-31}

\begin{document}
\maketitle

\subsection{Download data}\label{download-data}

Instruction:

\begin{enumerate}
\def\labelenumi{\arabic{enumi}.}
\item
  Go to \includegraphics{https://nces.ed.gov/ipeds/use-the-data}, and
  click on \textbf{Complete Data Files}
\item
  Download Data File HD2024, EF2024D, etc.
\item
  Open the zipped file, extract and put the raw csv into this repo's
  data/raw/ folder
\end{enumerate}

\begin{Shaded}
\begin{Highlighting}[]
\CommentTok{\# Directory}
\NormalTok{hd2024 }\OtherTok{\textless{}{-}} \FunctionTok{read.csv}\NormalTok{(}\StringTok{"data/raw/hd2024.csv"}\NormalTok{)}

\CommentTok{\# Fall Enrollment}
\NormalTok{ef2024d }\OtherTok{\textless{}{-}} \FunctionTok{read.csv}\NormalTok{(}\StringTok{"data/raw/ef2024d.csv"}\NormalTok{)}
\NormalTok{ef2024b }\OtherTok{\textless{}{-}} \FunctionTok{read.csv}\NormalTok{(}\StringTok{"data/raw/ef2024b.csv"}\NormalTok{)}

\CommentTok{\# Check dim}
\FunctionTok{dim}\NormalTok{(hd2024)}
\end{Highlighting}
\end{Shaded}

\begin{verbatim}
## [1] 6072   72
\end{verbatim}

\begin{Shaded}
\begin{Highlighting}[]
\FunctionTok{dim}\NormalTok{(ef2024d)}
\end{Highlighting}
\end{Shaded}

\begin{verbatim}
## [1] 5578   33
\end{verbatim}

\begin{Shaded}
\begin{Highlighting}[]
\FunctionTok{dim}\NormalTok{(ef2024b)}
\end{Highlighting}
\end{Shaded}

\begin{verbatim}
## [1] 76359    22
\end{verbatim}

\subsection{Process tables with multiple
id}\label{process-tables-with-multiple-id}

\begin{Shaded}
\begin{Highlighting}[]
\NormalTok{ef2024b\_inst }\OtherTok{\textless{}{-}}\NormalTok{ ef2024b }\SpecialCharTok{\%\textgreater{}\%}
  \FunctionTok{group\_by}\NormalTok{(UNITID) }\SpecialCharTok{\%\textgreater{}\%}
  \FunctionTok{summarize}\NormalTok{(}
    \AttributeTok{total\_enrollment =} \FunctionTok{sum}\NormalTok{(EFAGE09, }\AttributeTok{na.rm =} \ConstantTok{TRUE}\NormalTok{),}
    \AttributeTok{total\_men =} \FunctionTok{sum}\NormalTok{(EFAGE07, }\AttributeTok{na.rm =} \ConstantTok{TRUE}\NormalTok{),}
    \AttributeTok{total\_women =} \FunctionTok{sum}\NormalTok{(EFAGE08, }\AttributeTok{na.rm =} \ConstantTok{TRUE}\NormalTok{)}
\NormalTok{  )}
\end{Highlighting}
\end{Shaded}

\subsection{Get removed UNITIDs}\label{get-removed-unitids}

\begin{Shaded}
\begin{Highlighting}[]
\NormalTok{remove\_unitids }\OtherTok{\textless{}{-}} \FunctionTok{bind\_rows}\NormalTok{(}
  \CommentTok{\# 2{-}year schools}
\NormalTok{  hd2024 }\SpecialCharTok{\%\textgreater{}\%}
    \FunctionTok{filter}\NormalTok{(SECTOR }\SpecialCharTok{\%in\%} \FunctionTok{c}\NormalTok{(}\DecValTok{4}\NormalTok{, }\DecValTok{5}\NormalTok{, }\DecValTok{6}\NormalTok{, }\DecValTok{7}\NormalTok{, }\DecValTok{8}\NormalTok{, }\DecValTok{9}\NormalTok{, }\DecValTok{99}\NormalTok{)) }\SpecialCharTok{\%\textgreater{}\%}
    \FunctionTok{select}\NormalTok{(UNITID),}
  
  \CommentTok{\# \textless{} 500 population}
\NormalTok{  ef2024b\_inst }\SpecialCharTok{\%\textgreater{}\%}
    \FunctionTok{filter}\NormalTok{(total\_enrollment }\SpecialCharTok{\textless{}} \DecValTok{500} \SpecialCharTok{|} \FunctionTok{is.na}\NormalTok{(total\_enrollment)) }\SpecialCharTok{\%\textgreater{}\%}
    \FunctionTok{select}\NormalTok{(UNITID)}
\NormalTok{) }\SpecialCharTok{\%\textgreater{}\%}
  \FunctionTok{distinct}\NormalTok{()}
\end{Highlighting}
\end{Shaded}

\subsection{Helper function to remove
rows}\label{helper-function-to-remove-rows}

\begin{Shaded}
\begin{Highlighting}[]
\NormalTok{filter\_ipeds\_2024 }\OtherTok{\textless{}{-}} \ControlFlowTok{function}\NormalTok{(df, remove\_tbl) \{}
\NormalTok{  df }\SpecialCharTok{\%\textgreater{}\%}
    \FunctionTok{anti\_join}\NormalTok{(remove\_tbl, }\AttributeTok{by =} \StringTok{"UNITID"}\NormalTok{)}
\NormalTok{\}}
\end{Highlighting}
\end{Shaded}

\subsection{Remove rows and write
tables}\label{remove-rows-and-write-tables}

\begin{Shaded}
\begin{Highlighting}[]
\NormalTok{hd2024\_clean }\OtherTok{\textless{}{-}} \FunctionTok{filter\_ipeds\_2024}\NormalTok{(hd2024, remove\_unitids)}
\NormalTok{ef2024d\_clean }\OtherTok{\textless{}{-}} \FunctionTok{filter\_ipeds\_2024}\NormalTok{(ef2024d, remove\_unitids)}
\NormalTok{ef2024b\_clean }\OtherTok{\textless{}{-}} \FunctionTok{filter\_ipeds\_2024}\NormalTok{(ef2024b, remove\_unitids)}

\FunctionTok{dim}\NormalTok{(hd2024\_clean)}
\end{Highlighting}
\end{Shaded}

\begin{verbatim}
## [1] 2830   72
\end{verbatim}

\begin{Shaded}
\begin{Highlighting}[]
\FunctionTok{dim}\NormalTok{(ef2024d\_clean)}
\end{Highlighting}
\end{Shaded}

\begin{verbatim}
## [1] 2467   33
\end{verbatim}

\begin{Shaded}
\begin{Highlighting}[]
\FunctionTok{dim}\NormalTok{(ef2024b\_clean)}
\end{Highlighting}
\end{Shaded}

\begin{verbatim}
## [1] 50290    22
\end{verbatim}

\begin{Shaded}
\begin{Highlighting}[]
\FunctionTok{write.csv}\NormalTok{(hd2024\_clean, }\StringTok{"data/cleaned/hd2024\_clean.csv"}\NormalTok{, }\AttributeTok{row.names =} \ConstantTok{FALSE}\NormalTok{)}
\FunctionTok{write.csv}\NormalTok{(ef2024d\_clean, }\StringTok{"data/cleaned/ef2024d\_clean.csv"}\NormalTok{, }\AttributeTok{row.names =} \ConstantTok{FALSE}\NormalTok{)}
\FunctionTok{write.csv}\NormalTok{(ef2024b\_clean, }\StringTok{"data/cleaned/ef2024b\_clean.csv"}\NormalTok{, }\AttributeTok{row.names =} \ConstantTok{FALSE}\NormalTok{)}
\end{Highlighting}
\end{Shaded}


\end{document}
